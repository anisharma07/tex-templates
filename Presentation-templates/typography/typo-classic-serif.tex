\documentclass[aspectratio=169, 11pt]{beamer}
\usetheme{default}
\usecolortheme{default}

% Typography: Classic Serif (Palatino)
\usefonttheme{serif}
\usepackage{palatino} % Use Palatino font
\usepackage{eulervm}  % Use Euler math fonts which go well with Palatino

% Packages
\usepackage{amsmath,amssymb,amsthm}
\usepackage{graphicx}
\usepackage{listings}
\usepackage{xcolor}

% Color scheme: Classic Black & Red
\definecolor{classicred}{RGB}{150, 0, 0}
\definecolor{textblack}{RGB}{20, 20, 20}

\setbeamercolor{frametitle}{fg=classicred,bg=white}
\setbeamerfont{frametitle}{family=\rmfamily, series=\bfseries, size=\Large}
\setbeamercolor{title}{fg=classicred}
\setbeamerfont{title}{family=\rmfamily, series=\bfseries, size=\huge}
\setbeamercolor{structure}{fg=classicred}
\setbeamercolor{normal text}{fg=textblack,bg=white}

% Serif blocks
\setbeamerfont{block title}{family=\rmfamily, series=\bfseries}
\setbeamercolor{block title}{bg=classicred!10, fg=classicred}
\setbeamercolor{block body}{bg=classicred!5}

\setbeamertemplate{navigation symbols}{}
\setbeamertemplate{footline}[frame number]

% Title Information
\title{Classic Serif Theme}
\subtitle{Traditional Academic Typography}
\author{Author Name}
\date{\today}

\begin{document}

\begin{frame}
\titlepage
\end{frame}

\begin{frame}{Table of Contents}
\tableofcontents
\end{frame}

\section{Section Name}

\begin{frame}{Standard Slide}
\begin{itemize}
\item First point with \textbf{bold emphasis}
\item Second point with \textit{italic emphasis}
\item Third point
\end{itemize}
\end{frame}

\begin{frame}{Slide with Block}
\begin{block}{Block Title}
This is a standard block useful for definitions or theorems.
\end{block}

\begin{alertblock}{Alert Block}
This is an alert block useful for warnings or important notes.
\end{alertblock}

\begin{exampleblock}{Example Block}
This is an example block useful for examples.
\end{exampleblock}
\end{frame}

\begin{frame}{Two Column Layout}
\begin{columns}
\begin{column}{0.5\textwidth}
\textbf{Left Column}
\begin{itemize}
\item Item 1
\item Item 2
\end{itemize}
\end{column}

\begin{column}{0.5\textwidth}
\textbf{Right Column}
\begin{center}
% \includegraphics[width=0.8\textwidth]{image-file}
[Image Placeholder]
\end{center}
\end{column}
\end{columns}
\end{frame}

\section{Mathematics}

\begin{frame}{Mathematical Content}
Equation example:
$$ E = mc^2 $$

Align environment example:
\begin{align*}
f(x) &= x^2 + 2x + 1 \\
&= (x+1)^2
\end{align*}
\end{frame}

\begin{frame}{Summary}
\begin{itemize}
\item Summary point 1
\item Summary point 2
\end{itemize}
\end{frame}

\end{document}
