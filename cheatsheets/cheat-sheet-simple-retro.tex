\documentclass[a4paper,8pt]{article}
\usepackage[landscape, margin=0.5in]{geometry}
\usepackage{multicol}
\usepackage{amsmath}
\usepackage{booktabs}
\usepackage{titlesec}
\usepackage{xcolor}
\usepackage{pagecolor}

\definecolor{termbg}{RGB}{10, 10, 10}
\definecolor{termfg}{RGB}{0, 255, 65} 
\definecolor{termheader}{RGB}{255, 255, 0}

\pagecolor{termbg}
\color{termfg}
\renewcommand{\familydefault}{\ttdefault}

\titleformat{\section}
{\color{termheader}\bfseries}{> }{0em}{}
\titlespacing*{\section}{0pt}{10pt}{2pt}

\setlength{\parindent}{0pt}
\setlength{\parskip}{0.5em}

\pagestyle{empty}

\begin{document}

\begin{center}
    {\large \textbf{user@host:~\$ cat complexity\_cheatsheet.txt}}
\end{center}

\begin{multicols}{3}

\section{Definitions}
\textbf{[Theta] Tight Bound}
$0 \le c_1 g(n) \le f(n) \le c_2 g(n)$
for $n \ge n_0$.

\textbf{[Big-O] Upper Bound}
$0 \le f(n) \le c g(n)$
for $n \ge n_0$.

\textbf{[Omega] Lower Bound}
$0 \le c g(n) \le f(n)$
for $n \ge n_0$.

\textbf{[Little-o] Strict Upper}
$\lim_{n \to \infty} \frac{f(n)}{g(n)} = 0$

\textbf{[Little-omega] Strict Lower}
$\lim_{n \to \infty} \frac{f(n)}{g(n)} = \infty$

\section{Standard Functions}
\textbf{Polynomials}: $\sum a_i n^i = \Theta(n^d)$

\textbf{Logarithms}: $\log_b n = \Theta(\ln n)$

\textbf{Factorials}: $n! = o(n^n)$

\section{Limits \& Comparisons}
$$ \lim_{n \to \infty} \frac{n^b}{a^n} = 0 \quad (\forall a > 1) $$
Exp functions $>$ Poly functions.

$$ \lim_{n \to \infty} \frac{\lg^b n}{n^a} = 0 \quad (\forall a > 0) $$
Poly functions $>$ Log functions.

\section{Properties}
\textbf{Transitivity}:
$f=\Theta(g), g=\Theta(h) \implies f=\Theta(h)$
(Applies to all).

\textbf{Transpose Symmetry}:
$f=O(g) \iff g=\Omega(f)$
$f=o(g) \iff g=\omega(f)$

\textbf{Equation Arithmetic}:
$2n^2 + 3n + 1 = \Theta(n^2)$

\section{Stirling's Approx}
$$ n! \approx \sqrt{2\pi n} \left(\frac{n}{e}\right)^n $$
Useful for analyzing factorial complexities.

\section{Recurrences (Master)}
$T(n) = aT(n/b) + f(n)$
Depending on $c = \log_b a$:
1. $f(n) = O(n^{c-\epsilon}) \implies \Theta(n^c)$
2. $f(n) = \Theta(n^c) \implies \Theta(n^c \lg n)$
3. $f(n) = \Omega(n^{c+\epsilon}) \implies \Theta(f(n))$

\section{Analytic Tricks}
\begin{itemize}
    \item Ignore constants.
    \item Ignore lower order terms.
    \item $n!$ grows VERY fast.
    \item $\lg^* n$ (Iterated log) grows VERY slow.
\end{itemize}

\end{multicols}
\end{document}
