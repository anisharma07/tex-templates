\documentclass[9pt,a4paper]{article}
\usepackage[margin=0.5in]{geometry}
\usepackage{amsmath,amssymb}
\usepackage{multicol}
\usepackage{graphicx}
\usepackage{xcolor}
\usepackage{titlesec}
\usepackage{booktabs}

% Use Charter or similar serif
\usepackage{charter}
\usepackage[T1]{fontenc}

\setlength{\parindent}{0pt}
\setlength{\parskip}{2pt}
\setlength{\columnsep}{1.5em}
\setlength{\columnseprule}{0pt}

% Elegant headers: Centered with lines
\newcommand{\sectiontitle}[1]{
    \vspace{8pt}
    \begin{center}
        \textbf{\large \textsc{#1}} \\
        \vspace{2pt}
        \rule{0.6\linewidth}{0.5pt}
    \end{center}
    \vspace{2pt}
}

\newcommand{\subsectiontitle}[1]{\textbf{\textit{#1}}}

\pagestyle{empty}

\begin{document}

\begin{center}
    \huge \textbf{Analysis of Algorithms} \\
    \vspace{5pt}
    \normalsize \textit{Quick Reference Guide}
\end{center}

\vspace{10pt}

\begin{multicols}{2}

\sectiontitle{Notations}

\begin{center}
\begin{tabular}{c l c}
\toprule
\textbf{Symbol} & \textbf{Concept} & \textbf{Definition} \\
\midrule
$\Theta$ & Tight Bound & $c_1 g \le f \le c_2 g$ \\
$O$ & Upper Bound & $f \le c g$ \\
$\Omega$ & Lower Bound & $c g \le f$ \\
$o$ & Strict Upper & $f < c g$ \\
$\omega$ & Strict Lower & $c g < f$ \\
\bottomrule
\end{tabular}
\end{center}

\vspace{5pt}
\subsectiontitle{Interpretations}
\begin{itemize}
    \item $O(g(n))$: $f$ grows no faster than $g$.
    \item $\Omega(g(n))$: $f$ grows at least as fast as $g$.
    \item $\Theta(g(n))$: $f$ grows at the same rate as $g$.
\end{itemize}

\sectiontitle{Complexity Classes}

\subsectiontitle{Constant $O(1)$}
Operations independent of input size. 
\textit{Ex: Hash table lookup, Array index.}

\subsectiontitle{Logarithmic $O(\lg n)$}
Problem size reduced by constant fraction.
\textit{Ex: Binary Search.}

\subsectiontitle{Linear $O(n)$}
Processing each element once.
\textit{Ex: Linear Search, Iteration.}

\subsectiontitle{Linearithmic $O(n \lg n)$}
Divide and conquer algorithms.
\textit{Ex: Merge Sort, Heap Sort.}

\subsectiontitle{Quadratic $O(n^2)$}
Nested loops over data.
\textit{Ex: Bubble Sort, Matrix Add.}

\sectiontitle{Fundamental Laws}

\subsectiontitle{Transitivity}
If $f(n) \in O(g(n))$ and $g(n) \in O(h(n))$, then $f(n) \in O(h(n))$.

\subsectiontitle{Symmetry}
$f(n) \in \Theta(g(n))$ if and only if $g(n) \in \Theta(f(n))$.

\subsectiontitle{Rule of Sums}
$O(f(n) + g(n)) = O(\max(f(n), g(n)))$.

\columnbreak

\sectiontitle{Limit Comparisons}

To compare $f(n)$ and $g(n)$, evaluate:
$$ L = \lim_{n \to \infty} \frac{f(n)}{g(n)} $$

\begin{itemize}
    \item If $L = 0$: $f \in o(g)$ (f is smaller)
    \item If $0 < L < \infty$: $f \in \Theta(g)$ (Same growth)
    \item If $L = \infty$: $f \in \omega(g)$ (f is larger)
\end{itemize}

\sectiontitle{Mathematical Identities}

\subsectiontitle{Logarithms}
$a = b^{\log_b a}$ \\
$\log_b a = \frac{\ln a}{\ln b}$ \\
$\log (n!) = \Theta(n \lg n)$

\subsectiontitle{Exponentials}
$a^{m+n} = a^m a^n$ \\
$(a^m)^n = a^{mn}$ \\
$n^b \ll a^n$ for any $b, a > 1$.

\subsectiontitle{Series}
$\displaystyle \sum_{i=1}^n i = \frac{n(n+1)}{2} \approx \frac{n^2}{2}$ \\
$\displaystyle \sum_{i=0}^n x^i = \frac{x^{n+1}-1}{x-1}$

\sectiontitle{The Master Theorem}

Used to solve recurrences of the form:
$$ T(n) = aT(n/b) + f(n) $$
\begin{enumerate}
    \item If $f(n)$ is polynomially smaller than $n^{\log_b a}$, then $T(n) = \Theta(n^{\log_b a})$.
    \item If $f(n)$ is approx equal to $n^{\log_b a}$, then $T(n) = \Theta(n^{\log_b a} \lg n)$.
    \item If $f(n)$ is polynomially larger, then $T(n) = \Theta(f(n))$.
\end{enumerate}

\sectiontitle{Guidelines}

\textit{Drop constants.} Efficiency scales with N. \\
\textit{Dominant terms only.} $n^2 + n \approx n^2$. \\
\textit{Base is irrelevant.} $\log_2 n \approx \log_{10} n$.

\end{multicols}
\end{document}
