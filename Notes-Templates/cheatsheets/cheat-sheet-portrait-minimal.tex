\documentclass[9pt,a4paper]{article}
\usepackage[margin=0.5in]{geometry}
\usepackage{amsmath,amssymb}
\usepackage{multicol}
\usepackage{graphicx}
\usepackage{enumitem}
\usepackage{booktabs}
\usepackage{mathpazo} % Palatino
\usepackage{titlesec}

\setlength{\parindent}{0pt}
\setlength{\parskip}{3pt}
\setlength{\columnseprule}{0.4pt}

% Minimalist Headers
\titleformat{\section}
  {\large\bfseries\scshape}{}{0em}{}
\titlespacing*{\section}{0pt}{8pt}{4pt}

\newcommand{\sectiontitle}[1]{\section*{#1}\hrule}
\newcommand{\subsectiontitle}[1]{\textbf{#1:}}

\pagestyle{empty}

\begin{document}

\begin{center}
    {\LARGE \textbf{Asymptotic Notation Cheat Sheet}} \\
    \vspace{4pt}
    \textit{Growth of Functions - Quick Reference (Portrait Minimal)}
\end{center}

\vspace{5pt}

\begin{multicols}{2}

\sectiontitle{The 5 Notations}

\begin{center}
\begin{tabular}{clc}
\toprule
\textbf{Not.} & \textbf{Meaning} & \textbf{Condition} \\
\midrule
$\Theta$ & Tight bound & $c_1 g \le f \le c_2 g$ \\
$O$ & Upper bound & $f \le c g$ \\
$\Omega$ & Lower bound & $c g \le f$ \\
$o$ & Strict upper & $f < c g$ \\
$\omega$ & Strict lower & $c g < f$ \\
\bottomrule
\end{tabular}
\end{center}

\vspace{5pt}
\subsectiontitle{$\Theta(g(n))$} $\exists c_1, c_2, n_0 > 0:$ $0 \leq c_1 g(n) \leq f(n) \leq c_2 g(n)$ $\forall n \geq n_0$

\vspace{3pt}
\subsectiontitle{$O(g(n))$} $\exists c, n_0 > 0:$ $0 \leq f(n) \leq c \cdot g(n)$ $\forall n \geq n_0$

\vspace{3pt}
\subsectiontitle{$\Omega(g(n))$} $\exists c, n_0 > 0:$ $0 \leq c \cdot g(n) \leq f(n)$ $\forall n \geq n_0$

\vspace{3pt}
\subsectiontitle{$o(g(n))$} $\lim_{n \to \infty} \frac{f(n)}{g(n)} = 0$

\vspace{3pt}
\subsectiontitle{$\omega(g(n))$} $\lim_{n \to \infty} \frac{f(n)}{g(n)} = \infty$

\sectiontitle{Key Theorems}

\textbf{Th 3.1:} $f(n) = \Theta(g(n)) \iff f(n) = O(g(n))$ AND $f(n) = \Omega(g(n))$

\textbf{Polynomials:} $p(n) = \sum_{i=0}^d a_i n^i \implies \Theta(n^d)$

\textbf{Transpose Terminology:} 
$f(n) = O(g(n)) \iff g(n) = \Omega(f(n))$

\sectiontitle{Growth Hierarchy}

$$O(1) < O(\lg n) < O(\sqrt{n}) < O(n)$$
$$< O(n \lg n) < O(n^2) < O(n^3)$$
$$< O(2^n) < O(n!)$$

\sectiontitle{Common Complexities}

\begin{center}
\begin{tabular}{ll}
\toprule
\textbf{Class} & \textbf{Algorithm} \\
\midrule
$O(1)$ & Stack op, Hash map \\
$O(\lg n)$ & Binary search \\
$O(n)$ & Linear scan \\
$O(n \lg n)$ & Optimal sort \\
$O(n^2)$ & Nested loops \\
$O(2^n)$ & Subsets \\
$O(n!)$ & Permutations \\
\bottomrule
\end{tabular}
\end{center}

\sectiontitle{Properties}

\textbf{Transitivity:} $f = \Theta(g), g = \Theta(h) \implies f = \Theta(h)$

\textbf{Reflexivity:} $f(n) = \Theta(f(n))$

\textbf{Symmetry:} $f(n) = \Theta(g(n)) \iff g(n) = \Theta(f(n))$

\columnbreak

\sectiontitle{Critical Limits}

\textbf{Exp vs Poly:} $\lim_{n \to \infty} \frac{n^b}{a^n} = 0 \quad (a > 1)$
$\implies n^b = o(a^n)$

\textbf{Poly vs Log:} $\lim_{n \to \infty} \frac{\lg^b n}{n^a} = 0 \quad (a > 0)$
$\implies \lg^b n = o(n^a)$

\textbf{Factorial vs Exp:} $n! = \omega(2^n)$

\sectiontitle{Logarithms \& Exponentials}

$\lg n = \log_2 n$, $\ln n = \log_e n$

\textbf{Log Identities:}
\begin{align*}
a &= b^{\log_b a} & \log(ab) &= \log a + \log b \\
\log(a^k) &= k \log a & \log_b a &= \frac{\log a}{\log b}
\end{align*}

\textbf{Exp Identities:}
$(a^m)^n = a^{mn}$, \quad $e^x \approx 1+x$

\sectiontitle{Factorials \& Stirling}

$n! = 1 \cdot 2 \cdot 3 \cdots n$

\textbf{Stirling's Approximation (Weak):}
$n! \approx (n/e)^n$

\textbf{Log Factorial:} $\lg(n!) = \Theta(n \lg n)$

\sectiontitle{Master Theorem}

For $T(n) = aT(n/b) + f(n)$: Let $c_{crit} = \log_b a$.

\textbf{1.} $f(n) = O(n^c)$ ($c < c_{crit}$) $\implies \Theta(n^{c_{crit}})$

\textbf{2.} $f(n) = \Theta(n^{c_{crit}})$ $\implies \Theta(n^{c_{crit}} \lg n)$

\textbf{3.} $f(n) = \Omega(n^c)$ ($c > c_{crit}$) $\implies \Theta(f(n))$

\sectiontitle{Loop Analysis}

\textbf{Single loop:} $O(n)$

\textbf{Nested (independent):} $O(nm)$

\textbf{Nested (dependent):} $\sum_{i=1}^n i = \Theta(n^2)$

\textbf{Halving loop:} $O(\lg n)$

\sectiontitle{Summation Formulas}

$\sum_{i=1}^n 1 = n$

$\sum_{i=1}^n i = \Theta(n^2)$

$\sum_{i=1}^n i^2 = \Theta(n^3)$

$\sum_{i=0}^n a^i = \frac{a^{n+1}-1}{a-1}$

\end{multicols}
\end{document}
