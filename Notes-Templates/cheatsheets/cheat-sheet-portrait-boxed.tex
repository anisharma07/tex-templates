\documentclass[9pt,a4paper]{article}
\usepackage[margin=0.5in]{geometry}
\usepackage{amsmath,amssymb}
\usepackage{multicol}
\usepackage{graphicx}
\usepackage{xcolor}
\usepackage{tgheros}
\renewcommand{\familydefault}{\sfdefault}
\usepackage{enumitem}
\usepackage[most]{tcolorbox}

\definecolor{headerbg}{RGB}{240, 240, 240}
\definecolor{boxborder}{RGB}{180, 180, 180}

\setlength{\parindent}{0pt}
\setlength{\parskip}{4pt}
\setlength{\columnsep}{1em}
\setlength{\columnseprule}{0pt}

% Boxed Section Style
\newcommand{\sectiontitle}[1]{
    \vspace{6pt}
    \begin{tcolorbox}[
        colback=headerbg,
        colframe=boxborder,
        boxrule=0.5pt,
        arc=2pt,
        left=2pt,right=2pt,top=2pt,bottom=2pt,
        fontupper=\bfseries\small\MakeUppercase
    ]
    #1
    \end{tcolorbox}
    \vspace{-2pt}
}

\newcommand{\subsectiontitle}[1]{\textbf{#1:}}

\pagestyle{empty}

\begin{document}

\begin{center}
    {\huge \textbf{Complexity Cheat Sheet}} \\
    \vspace{2pt}
    \small \textsc{Asymptotic Analysis Reference - Portrait Boxed}
\end{center}

\vspace{5pt}

\begin{multicols}{2}

\sectiontitle{The 5 Notations}
\begin{tcolorbox}[colback=white, colframe=white, boxrule=0pt, left=0pt, right=0pt, top=0pt, bottom=0pt]
\begin{tabular}{cll}
\textbf{Sym} & \textbf{Term} & \textbf{Analogy} \\
\hline
$\Theta$ & Tight & $a=b$ \\
$O$ & Upper & $a \le b$ \\
$\Omega$ & Lower & $a \ge b$ \\
$o$ & Strict Upper & $a < b$ \\
$\omega$ & Strict Lower & $a > b$ \\
\end{tabular}
\end{tcolorbox}

\sectiontitle{Definitions}
\subsectiontitle{$\Theta(g(n))$} $0 \leq c_1 g(n) \leq f(n) \leq c_2 g(n)$

\subsectiontitle{$O(g(n))$} $0 \leq f(n) \leq c \cdot g(n)$

\subsectiontitle{$\Omega(g(n))$} $0 \leq c \cdot g(n) \leq f(n)$

\subsectiontitle{$o(g(n))$} $\lim (f/g) = 0$

\subsectiontitle{$\omega(g(n))$} $\lim (f/g) = \infty$

\sectiontitle{Growth Hierarchy}
\begin{tcolorbox}[colback=yellow!10, colframe=white, boxrule=0pt]
$O(1) < O(\lg n) < O(\sqrt{n}) < O(n)$
$< O(n \lg n) < O(n^2) < O(2^n)$
\end{tcolorbox}

\sectiontitle{Common Complexities}
\begin{tabular}{ll}
$O(1)$ & Array access \\
$O(\lg n)$ & Binary Search \\
$O(n)$ & Linear Scan \\
$O(n \lg n)$ & Merge Sort \\
$O(n^2)$ & Bubble Sort \\
$O(2^n)$ & Power Set \\
\end{tabular}

\sectiontitle{Key Theorems}
\begin{itemize}[leftmargin=*, nosep]
    \item $f=\Theta(g) \iff O(g) \land \Omega(g)$
    \item Polynomials: Degree determines growh.
    \item Transpose: $f=O(g) \iff g=\Omega(f)$.
\end{itemize}

\columnbreak

\sectiontitle{Properties}
\textbf{Transitivity}: $f=\Theta(g) \land g=\Theta(h) \implies f=\Theta(h)$.
\textbf{Additivity}: $f+g = \max(f, g)$.

\sectiontitle{Critical Limits}
\textbf{Exp vs Poly}: $n^b = o(a^n)$ ($a>1$).
\textbf{Poly vs Log}: $\lg^b n = o(n^a)$ ($a>0$).
\textbf{Factorial}: $n!$ beats $2^n$ but $n! < n^n$.

\sectiontitle{Logarithms}
\begin{itemize}[leftmargin=*, nosep]
    \item $\log(ab) = \log a + \log b$
    \item $\log(a^k) = k \log a$
    \item Change of base: $\log_b a = \frac{\ln a}{\ln b}$
    \item Base doesn't matter for Big-O!
\end{itemize}

\sectiontitle{Master Theorem}
$T(n) = aT(n/b) + f(n)$, $c_{crit} = \log_b a$.
\begin{enumerate}[leftmargin=*, nosep]
    \item $f(n)$ implies $O(n^{c-\epsilon}) \to T(n) = \Theta(n^{c_{crit}})$
    \item $f(n)$ implies $\Theta(n^{c_{crit}}) \to T(n) = \Theta(n^{c_{crit}} \lg n)$
    \item $f(n)$ implies $\Omega(n^{c+\epsilon}) \to T(n) = \Theta(f(n))$
\end{enumerate}

\sectiontitle{Common Mistakes}
\begin{tcolorbox}[colback=red!5, colframe=red!20, title=Avoid These errors, fonttitle=\bfseries\small]
\small
[X] Saying "at least O(n)"
[X] Confusing $2^n$ with $n^2$
[X] Counting input size in space complexity
\end{tcolorbox}

\sectiontitle{Loop Analysis}
\begin{itemize}[leftmargin=*, nosep]
    \item \textbf{Linear}: for i:1..n $\to O(n)$
    \item \textbf{Nested}: for i:1..n, j:1..n $\to O(n^2)$
    \item \textbf{Dependent}: $\sum i \to O(n^2)$
    \item \textbf{Halving}: while n>1 (n/=2) $\to O(\lg n)$
\end{itemize}

\sectiontitle{Useful Series}
$\sum_{i=1}^n i = \frac{n(n+1)}{2} = \Theta(n^2)$

$\sum_{i=0}^n r^i = \frac{r^{n+1}-1}{r-1}$ (Geom)

\end{multicols}
\end{document}
