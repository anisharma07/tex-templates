\documentclass[aspectratio=169, 12pt]{beamer}
\usetheme{default}
\usecolortheme{default}

% Typography: Modern Sans Serif
% Using default sans serif but with structure adjustments
\usefonttheme{professionalfonts} % Prevents beamer from overwriting some font settings
\usepackage{helvet} % Use Helvetica as sans serif
\renewcommand{\familydefault}{\sfdefault}

% Packages
\usepackage{amsmath,amssymb,amsthm}
\usepackage{graphicx}
\usepackage{listings}
\usepackage{xcolor}

% Color scheme: Clean Grey/Blue
\definecolor{modernblue}{RGB}{0, 102, 204}
\definecolor{darkgrey}{RGB}{64, 64, 64}

\setbeamercolor{frametitle}{fg=modernblue,bg=white}
\setbeamerfont{frametitle}{series=\bfseries, size=\Large}
\setbeamercolor{title}{fg=modernblue}
\setbeamerfont{title}{series=\bfseries, size=\huge}
\setbeamercolor{structure}{fg=modernblue}
\setbeamercolor{normal text}{fg=darkgrey,bg=white}

% Bold itemize
\setbeamertemplate{itemize item}{\color{modernblue}$\blacksquare$}
\setbeamertemplate{itemize subitem}{\color{modernblue}$\bullet$}
\setbeamerfont{itemize/enumerate body}{size=\large}
\setbeamerfont{itemize/enumerate subbody}{size=\normalsize}

\setbeamertemplate{navigation symbols}{}
\setbeamertemplate{footline}[frame number]

% Title Information
\title{Modern Sans Theme}
\subtitle{Clean & Readable Typography}
\author{Author Name}
\date{\today}

\begin{document}

\begin{frame}
\titlepage
\end{frame}

\begin{frame}{Table of Contents}
\tableofcontents
\end{frame}

\section{Section Name}

\begin{frame}{Standard Slide}
\begin{itemize}
\item First point with \textbf{bold emphasis}
\item Second point with \textit{italic emphasis}
\item Third point
\end{itemize}
\end{frame}

\begin{frame}{Slide with Block}
\begin{block}{Block Title}
This is a standard block useful for definitions or theorems.
\end{block}

\begin{alertblock}{Alert Block}
This is an alert block useful for warnings or important notes.
\end{alertblock}

\begin{exampleblock}{Example Block}
This is an example block useful for examples.
\end{exampleblock}
\end{frame}

\begin{frame}{Two Column Layout}
\begin{columns}
\begin{column}{0.5\textwidth}
\textbf{Left Column}
\begin{itemize}
\item Item 1
\item Item 2
\end{itemize}
\end{column}

\begin{column}{0.5\textwidth}
\textbf{Right Column}
\begin{center}
% \includegraphics[width=0.8\textwidth]{image-file}
[Image Placeholder]
\end{center}
\end{column}
\end{columns}
\end{frame}

\section{Mathematics}

\begin{frame}{Mathematical Content}
Equation example:
$$ E = mc^2 $$

Align environment example:
\begin{align*}
f(x) &= x^2 + 2x + 1 \\
&= (x+1)^2
\end{align*}
\end{frame}

\begin{frame}{Summary}
\begin{itemize}
\item Summary point 1
\item Summary point 2
\end{itemize}
\end{frame}

\end{document}
